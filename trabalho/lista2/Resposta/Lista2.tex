\documentclass{article}
\usepackage{graphicx}
\usepackage{color,soul,listings}
\usepackage{framed}
\usepackage{xcolor}

\begin{document} \begin{figure}
\centering
    \includegraphics[scale=0.3]{./LogoUFRJ.png }
\caption{Logo Ufrj}
\end{figure}

\title{Lista 2 computacao concorrente}
\author{Lucas Tatsuya Tanaka\\
    \large Prof:Silvana Rosseto\\
    \large Dre:118058149\\
}
\maketitle

\definecolor{shadecolor}{gray}{0.9}
\definecolor{dkgreen}{rgb}{0,0.6,0}
\definecolor{gray}{rgb}{0.5,0.5,0.5}
\definecolor{mauve}{rgb}{0.58,0,0.82}

\lstset{frame=tb,
  language=Java,
  aboveskip=3mm,
  belowskip=3mm,
  showstringspaces=false,
  columns=flexible,
  basicstyle={\small\ttfamily},
  numbers=none,
  numberstyle=\tiny\color{gray},
  keywordstyle=\color{blue},
  commentstyle=\color{dkgreen},
  stringstyle=\color{mauve},
  breaklines=true,
  breakatwhitespace=true,
  tabsize=3
}

\section{Questao 1}\label{questao-1}

\begin{itemize}
\item
  \textbf{A)} Apos a verificação do código se observa que sendo a
  principal condição do programa a alternância entre as Threads, se
  averiguá que com o início das Threads a Thread Foo tomara prioridade
  na execução devido à presença do pthread\_cond\_wait(\&condBar,
  \&mutex) que bloqueara as Threads Bar ate o termino de execução de
  todas as Threads Foo, apos o termino da execução a última Thread Foo
  ira adentrar o condicional e como contaFoo sera igual a M ocorrera o
  broadcast das Threads bloqueadas por condBar, ocorrera então a entrada
  no While da Thread Bar no qual ela se bloqueara ate que a última
  Thread adentre o condicional e desbloqueia as Threads bloqueadas. por
  condFoo e que assim por diante irão alternar a execução das Threads.
\item
  \textbf{B)} O código apresentados apos ser verificado, garante a
  inexistência de deadlocks e ausência de condição de corrida isto
  ocorre, pois, devido as condições do programa serem atendidas os
  recursos são acessados através de exclusão mutua e assim garantem os
  resultados esperados, com relação aos deadlocks não a possibilidade do
  mesmo, pois sempre que a última Thread chegar ao final da execução ele
  atendera a condição de ser igual ao M ou N e ira executar o broadcast
  liberando as Threads.
\end{itemize}

\section{Questao 2}\label{questao-2}

\begin{itemize}
\item
  \textbf{A)} O que causou o deadlock foi a situacão ao qual a Thread
  apresentava mais Threads remove do que as Threads insere
\item
  \textbf{B)} Uma possivel ordem de execucao seria o total bloqueio de
  todas as Threads Remove inicialmente e em seguida realizar uma Thread
  insere que vai desbloquear uma Thread remove e assim repedidamente
  caso existam mais Threads remove o programa entrara entrara em
  deadlock
\item
  \textbf{C)} Uma possivel correcao seria
\end{itemize}

\begin{shaded}
    \begin{lstlisting}[languge=java]
    class Buffer{
    static final int N = 10; //qtde de elementos no buffer
    private int[] buffer = new int[N]; //area de dados compartilhada
    private int count=0; //qtde de posicoes ocupadas no buffer
    private int in=0; //proxima posicao de insercao 
    private int out=0; //proxima posicao de retirada

    //construtot
    Buffer() {... //inicializa o buffer }

    // Insere um item
    public synchronized void Insere (int item) {
        while (count==N) {
            wait();
        }
        buffer[in%N] = item;
        in++;
        count++;
        notifyAll();
    }
    
    // Remove um item
    public synchronized int Remove (int id) {
        int aux;
        while (count==0) {
            wait();
        }
        aux = buffer[out%N];
        out++;
        count--;
        notifyAll();
        return aux;
    }
}

    \end{lstlisting}
\end{shaded}

\section{Questao 3}\label{questao-3}

\begin{itemize}
\tightlist
\item
  \textbf{A)} Apos a averiguacao do programa se entende que a Thread B
  avia se bloqueado e depois de 10 execucoes do While da Thread A a
  Thread b ganha execucao pois essa ja avia sido desbloqueada pelo
  signal anteriormente executado pela Thread A.
\item
  \textbf{B)} O codigo utilizado pode ser:
\end{itemize}

\begin{shaded}
    \begin{lstlisting}[language=c]
    int x = 0; pthread_mutex_t x_mutex; pthread_cond_t x_cond;
void *A (void *tid ){
    for (int i=0; i<100; i++) {
        pthread_mutex_lock(&x_mutex);
        x++;
        if(!(x%10))
            pthread_cond_signal(&x_cond);
        pthread_mutex_unlock(&x_mutex);
    }   
}

void *B (void *tid) {
    pthread_mutex_lock(&x_mutex);
    if(x%10){
        pthread_cond_wait(&x_cond, &x_mutex);
        printf("X=%d\n", x);
    }
    pthread_mutex_unlock(&x_mutex);
}

    \end{lstlisting}
\end{shaded}

\section{Questao 4}\label{questao-4}

\begin{itemize}
\tightlist
\item
  \textbf{A)} A solução proposta atende a todos os requisitos
  apresentados mais de um leitor pode realizar a leitura como averiguado
  no metodo EntraLEitor na qual nao a nenhum bloqueio, o segundo
  requisito é atendido quando analizado o synchronized do metodo Escrita
  na qual permite apenas a execucao de um dos metodos de cada vez e
  ocorre a garantia de que quando ocorre a escrita nao existe mais
  leitores pois se atende a condicao do While no metodo.
\item
  \textbf{B)} Apos a averiguação do programa se entende que com a troca
  do notifyAll por notify ocorre a garantia ainda da execucao do
  programa sem afetar a corretude da solução isso ocorrendo pois com a
  chegada na linha 11 o notify poderia estar em duas situacoes
  principais a primeira sendo o bloqueio de todas as Threads escritoras
  no momento em que o notify fosse executado a Thread escritor iria se
  liberar e assim poderia se executar e devido a presença do notify no
  final da Thread Escrita este iria liberar em efeito domino as Threads
  seguintes na segunda existe parcialidade ou nenhuma Thread bloqueada
  devido ao fato da Thread ter this.leit iguais a zero a Thread
  Ecritores não iria se bloquear e assim ocorreria a execução do
  programa.
\item
  \textbf{C)} A chamada do metodo notify() poderia ser retirada pois
  devido a presença do metodo notifyAll presente na Thread leitora todas
  as Threads qua aviam sido previamente bloqueadas seriam liberadas
  fazendo assim a retomada de execução das Threads Escriotras e assim
  terminando a execução do programa.
\end{itemize}

\end{document}

